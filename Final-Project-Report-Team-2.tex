\documentclass{article}
\usepackage[utf8]{inputenc}
\usepackage{pdfpages}
\usepackage{graphicx}
\usepackage{hyperref}
\usepackage{amsmath}
\usepackage{amsfonts}
\usepackage{amssymb}
\usepackage{natbib}
\usepackage{algorithm}
\usepackage{algorithmic}
\usepackage{booktabs}
\usepackage{longtable}
\usepackage{tabularx}
\usepackage{multirow}
\usepackage{subcaption}
\usepackage{float}

\title{Final Team Project Report}
\author{Team Members Names}
\date{Due Date}

\begin{document}

\includepdf[pages={1}]{cover-report.pdf}

\maketitle

\section{Introduction}

\label{sec:introduction}
% Briefly introduce the problem and the context. State the purpose, goals, and scope of your project.
- Briefly introduce the problem you are addressing in your project.
- Provide some context about why this problem is important.
- Clearly state the purpose of your project.
- Outline the goals you hope to achieve with your project.
- Define the scope of your project, including any limitations.

\section{Related Work}
\label{sec:related_work}
% Discuss related work and how your approach is different/better.
- Discuss previous work that has been done on this problem.
- Compare and contrast your approach with these previous methods.
- Explain why your approach is different or better.

\section{Problem Definition and AI Techniques}
\label{sec:problem_definition_and_ai_techniques}
% Define the problem and describe the AI techniques used in your project.
- Clearly define the problem you are addressing.
- Describe the AI techniques you used in your project.
- Explain why these techniques are appropriate for your problem.

\section{Dataset Description}
\label{sec:dataset_description}
% Describe the dataset used in your project.
- Describe the dataset you used in your project.
- Discuss any relevant characteristics of the data.
- Explain how the data was collected and any preprocessing steps you took.

\section{Experimental Design}
\label{sec:experimental_design}
% Describe the experimental design, including any preprocessing steps, feature engineering, and the specific AI techniques used.
- Describe the design of your experiment.
- Discuss any preprocessing steps you took, such as cleaning the data or
dealing with missing values.
- Explain any feature engineering you did, such as creating new variables or
transforming existing ones.
- Detail the specific AI techniques you used and why you chose them.

\section{Results and Discussion}
\label{sec:results_and_discussion}
% Present the results of your experiments and discuss their implications.
- Present the results of your experiments, including any relevant figures or
tables.
- Discuss the implications of your results.
- Explain whether your results support your initial hypothesis.

\section{Conclusion and Future Work}
\label{sec:conclusion_and_future_work}
% Conclude the report, summarise the findings and discuss potential future work.
- Summarise the main findings of your project.
- Conclude the report by discussing the significance of your findings.
- Discuss potential future work, such as how your project could be extended or
improved.

\section{Contributions}
\label{sec:contributions}
% List each team member and describe their contributions to the project.
- List each team member.
- Describe the contributions of each team member to the project.

\section{References}
\label{sec:references}
% Include any references cited in your report.
- Include any references you cited in your report.

\section{Appendix}
\label{sec:appendix}
% Include any additional information or material that supports your report, such as full-sized images or extensive code listings.
- Include any additional information or material that supports your report.
- This could include full-sized images, extensive code listings, or additional
data.

\end{document}
