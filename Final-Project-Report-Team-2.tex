\documentclass{article}
\usepackage{algorithm}
\usepackage{algorithmic}
\usepackage{amsfonts}
\usepackage{amsmath}
\usepackage{amssymb}
\usepackage{booktabs}
\usepackage{float}
\usepackage{graphicx}
\usepackage{hyperref}
\usepackage{lipsum}
\usepackage{longtable}
\usepackage{multirow}
\usepackage{natbib}
\usepackage{pdfpages}
\usepackage{setspace}
\usepackage{subcaption}
\usepackage{tabularx}
\usepackage{tikz}
\usepackage[utf8]{inputenc}
\begin{document}

\setstretch{1.15}
\definecolor{blue}{RGB}{0,62,126}

\newcommand{\mytitlepage}[2]{
    \thispagestyle{empty}

    \begin{tikzpicture}[remember picture, overlay]
        \node [inner sep=0pt] at (current page.center) {#1};
        {
        \node [
            anchor=center,
            inner sep=1.25cm,
            rectangle,
            fill=blue!70!white,
            fill opacity=0,
            text opacity=1,
            minimum height=0.2\paperheight,
            minimum width=\paperwidth,
            text width=0.8\paperwidth,
            font=\fontfamily{pnc}\selectfont
        ] at (current page.center) {#2};
        }
    \end{tikzpicture}

    \newpage
}

{
    \mytitlepage{\includegraphics[width=\paperwidth]{images/background.pdf}}
    {
        \centering\fontfamily{phv}\selectfont
        {\Huge \bfseries Attention Is All You Need \par}
        \vspace{8pt}
        {\begin{center}
                \begin{tabular*}{\textwidth}{@{\extracolsep{\fill}}c c c}
                    {\LARGE Jonathan Agustin} &
                    {\LARGE Fernando Calderon} &
                    {\LARGE Juliet Lawton}
                \end{tabular*}
            \end{center}}
        \vspace{24pt}
        {\LARGE\bfseries \par}
    }
}

\section{Introduction}
\begin{itemize}
    \item Briefly introduce the problem you are addressing in your project.
    \item Provide some context about why this problem is important.
    \item Clearly state the purpose of your project.
    \item Outline the goals you hope to achieve with your project.
    \item Define the scope of your project, including any limitations.
\end{itemize}

\section{Related Work}
\begin{itemize}
    \item Discuss previous work that has been done on this problem.
    \item Compare and contrast your approach with these previous methods.
    \item Explain why your approach is different or better.
\end{itemize}

\section{Problem Definition and AI Techniques}
\begin{itemize}
    \item Clearly define the problem you are addressing.
    \item Describe the AI techniques you used in your project.
    \item Explain why these techniques are appropriate for your problem.
\end{itemize}

\section{Dataset Description}
\begin{itemize}
    \item Describe the dataset you used in your project.
    \item Discuss any relevant characteristics of the data.
    \item Explain how the data was collected and any preprocessing steps you
          took.
\end{itemize}

\section{Experimental Design}
\begin{itemize}
    \item Describe the design of your experiment.
    \item Discuss any preprocessing steps you took, such as cleaning the data
          or
          dealing with missing values.
    \item Explain any feature engineering you did, such as creating new
          variables
          or transforming existing ones.
    \item Detail the specific AI techniques you used and why you chose them.
\end{itemize}

\section{Results and Discussion}
\begin{itemize}
    \item Present the results of your experiments, including any relevant
          figures
          or tables.
    \item Discuss the implications of your results.
    \item Explain whether your results support your initial hypothesis.
\end{itemize}

\section{Conclusion and Future Work}
\begin{itemize}
    \item Summarise the main findings of your project.
    \item Conclude the report by discussing the significance of your findings.
    \item Discuss potential future work, such as how your project could be
          extended
          or improved.
\end{itemize}

\section{Contributions}
\begin{itemize}
    \item List each team member.
    \item Describe the contributions of each team member to the project.
\end{itemize}

\section{References}
\begin{itemize}
    \item Include any references you cited in your report.
\end{itemize}

\section{Appendix}
\begin{itemize}
    \item Include any additional information or material that supports your
          report.
    \item This could include full-sized images, extensive code listings, or
          additional data.
\end{itemize}

\end{document}
