The goal of reducing sequential computation also forms the foundation of the
Extended Neural GPU \citep{extendedngpu}, ByteNet \citep{NalBytenet2017} and
ConvS2S \citep{JonasFaceNet2017}, all of which use convolutional neural
networks as basic building block, computing hidden representations in parallel
for all input and output positions. In these models, the number of operations
required to relate signals from two arbitrary input or output positions grows
in the distance between positions, linearly for ConvS2S and logarithmically for
ByteNet. This makes it more difficult to learn dependencies between distant
positions \citep{hochreiter2001gradient}. In the Transformer this is reduced to
a constant number of operations, albeit at the cost of reduced effective
resolution due to averaging attention-weighted positions, an effect we
counteract with Multi-Head Attention as described in
section~\ref{sec:attention}.

Self-attention, sometimes called intra-attention is an attention mechanism
relating different positions of a single sequence in order to compute a
representation of the sequence. Self-attention has been used successfully in a
variety of tasks including reading comprehension, abstractive summarization,
textual entailment and learning task-independent sentence representations
\citep{cheng2016long, decomposableAttnModel, paulus2017deep,
    lin2017structured}.

End-to-end memory networks are based on a recurrent attention mechanism instead
of sequence-aligned recurrence and have been shown to perform well on
simple-language question answering and language modeling tasks
\citep{sukhbaatar2015}.

To the best of our knowledge, however, the Transformer is the first
transduction model relying entirely on self-attention to compute
representations of its input and output without using sequence-aligned RNNs or
convolution.
In the following sections, we will describe the Transformer, motivate
self-attention and discuss its advantages over models such as
\citep{neural_gpu, NalBytenet2017} and \citep{JonasFaceNet2017}.

%\citep{JonasFaceNet2017} report new SOTA on machine translation for English-to-German (EnDe), Enlish-to-French (EnFr) and English-to-Romanian language pairs.

%For example,! in MT, we must draw information from both input and previous output words to translate an output word accurately. An attention layer \citep{bahdanau2014neural} can connect a very large number of positions at low computation cost, making it an essential ingredient in competitive recurrent models for machine translation.

%A natural question to ask then is, "Could we replace recurrence with attention?". \marginpar{Don't know if it's the most natural question to ask given the previous statements. Also, need to say that the complexity table summarizes these statements} Such a model would be blessed with the computational efficiency of attention and the power of cross-positional communication. In this work, show that pure attention models work remarkably well for MT, achieving new SOTA results on EnDe and EnFr, and can be trained in under $2$ days on xyz architecture.

%After the seminal models introduced in \citep{sutskever14, bahdanau2014neural, cho2014learning}, recurrent models have become the dominant solution for both sequence modeling and sequence-to-sequence transduction. Many efforts such as \citep{wu2016google,luong2015effective,jozefowicz2016exploring} have pushed the boundaries of machine translation (MT) and language modeling with recurrent endoder-decoder and recurrent language models. Recent effort \citep{shazeer2017outrageously} has successfully combined the power of conditional computation with sequence models to train very large models for MT, pushing SOTA at lower computational cost.

%Recurrent models compute a vector of hidden states $h_t$, for each time step $t$ of computation. $h_t$ is a function of both the input at time $t$ and the previous hidden state $h_t$. This dependence on the previous hidden state precludes processing all timesteps at once, instead requiring long sequences of sequential operations.  In practice, this results in greatly reduced computational efficiency, as on modern computing hardware, a single operation on a large batch is much faster than a large number of operations on small batches.  The problem gets worse at longer sequence lengths. Although sequential computation is not a severe bottleneck at inference time, as autoregressively generating each output requires all previous outputs, the inability to compute scores at all output positions at once hinders us from rapidly training our models over large datasets. Although impressive work such as \citep{Kuchaiev2017Factorization} is able to significantly accelerate the training of LSTMs with factorization tricks, we are still bound by the linear dependence on sequence length.

%If the model could compute hidden states at each time step using only the inputs and outputs,  it would be liberated from the dependence on results from previous time steps during training. This line of thought is the foundation of recent efforts such as the Markovian neural GPU \citep{neural_gpu}, ByteNet \citep{NalBytenet2017} and ConvS2S \citep{JonasFaceNet2017}, all of which use convolutional neural networks as a building block to compute hidden representations simultaneously for all timesteps, resulting in $O(1)$ sequential time complexity. \citep{JonasFaceNet2017} report new SOTA on machine translation for English-to-German (EnDe), Enlish-to-French (EnFr) and English-to-Romanian language pairs.

%A crucial component for accurate sequence prediction is modeling cross-positional communication. For example, in MT, we must draw information from both input and previous output words to translate an output word accurately. An attention layer \citep{bahdanau2014neural} can connect a very large number of positions at a low computation cost, also $O(1)$ sequential time complexity, making it an essential ingredient in recurrent encoder-decoder architectures for MT. A natural question to ask then is, "Could we replace recurrence with attention?". \marginpar{Don't know if it's the most natural question to ask given the previous statements. Also, need to say that the complexity table summarizes these statements} Such a model would be blessed with the computational efficiency of attention and the power of cross-positional communication. In this work, show that pure attention models work remarkably well for MT, achieving new SOTA results on EnDe and EnFr, and can be trained in under $2$ days on xyz architecture.

%Note: Facebook model is no better than RNNs in this regard, since it requires a number of layers proportional to the distance you want to communicate.  Bytenet is more promising, since it requires a logarithmnic number of layers (does bytenet have SOTA results)?

%Note: An attention  layer can connect a very large number of positions at a low computation cost in O(1) sequential operations.  This is why encoder-decoder attention has been so successful in seq-to-seq models so far.  It is only natural, then, to also use attention to connect the timesteps of the same sequence.

%Note: I wouldn't say that long sequences are not a problem during inference.  It would be great if we could infer with no long sequences.  We could just say later on that, while our training graph is constant-depth, our model still requires sequential operations in the decoder part during inference due to the autoregressive nature of the model.

%\begin{table}[h!]
%\caption{Attention models are quite efficient for cross-positional communications when sequence length is smaller than channel depth. $n$ represents the sequence length and $d$ represents the channel depth.}
%\label{tab:op_complexities}
%\begin{center}
%\vspace{-5pt}
%\scalebox{0.75}{

%\begin{tabular}{l|c|c|c}
%\hline \hline
%Layer Type & Receptive & Complexity & Sequential  \\
%           & Field     &            & Operations  \\
%\hline
%Pointwise Feed-Forward & $1$ & $O(n \cdot d^2)$ & $O(1)$ \\
%\hline
%Recurrent & $n$ & $O(n \cdot d^2)$ & $O(n)$ \\
%\hline
%Convolutional & $r$ & $O(r \cdot n \cdot d^2)$ & $O(1)$ \\
%\hline
%Convolutional (separable) & $r$ & $O(r \cdot n \cdot d + n %\cdot d^2)$ & $O(1)$ \\
%\hline
%Attention & $r$ & $O(r \cdot n \cdot d)$ & $O(1)$ \\
%\hline \hline
%\end{tabular}
%}
%\end{center}
%\end{table}
