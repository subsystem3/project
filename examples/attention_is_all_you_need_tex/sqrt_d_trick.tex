\section*{Justfication of the Scaling Factor in Dot-product Attention}

In Section~\ref{sec:scaled-dot-prod}, we introduced Scaled dot-product
attention, where we scale down the dot products by $\sqrt{d_k}$.   In this
section, we will give a rough justification of this scaling factor.  If we
assume that $q$ and $k$ are $d_k$-dimensional vectors whose components are
independent random variables with mean $0$ and variance $1$, then their dot
product, $q \cdot k = \sum_{i=1}^{d_k} u_iv_i$, has mean $0$ and variance
$d_k$.	Since we would prefer these values to have variance $1$, we divide by
$\sqrt{d_k}$.

%For any two $d_k$-dimension vectors $\vec{u}$ and $\vec{v}$, whose dimensions are independent, the mean and variance of the dot product will be the summation of the product of means and variances over the dimensions, that is, $E[<\vec{u},\vec{v}>] = \sum_{i=1}^{d_k} E[u_i]E[v_i]$, and $E[(<\vec{u},\vec{v}>-E[<\vec{u},\vec{v}>])^2] = \sum_{i=1}^{d_k} E[({u_i}-E[u_i])^2] E[({v_i}-E[v_i])^2]$. Layer norm encourages the mean and variance of each dimension to be $0$ and $1$ respectively, resultig in the dot product having mean $0$ and $d_k$ respectively. Therefore, scaling by $\sqrt{d_k}$ encourages the logits to be normalized as well.

\iffalse

    In this section, we will give a rough justification of this scaling factor,
    that is, we will show that for any two vectors, $\vec{u}$ and $\vec{v}$, whose
    variance and mean are $1$ and $0$ respectively, the variance and the mean of
    the dot product are $d_k$ and $0$ respectively. Therefore, dividing by
    $\sqrt{d_k}$ ensures that each component of the attention logits are
    normalized. The repeated layer norms at each transformer layer encourage
    $\vec{u}$ and $\vec{v}$ to be normalized.

    \begin{align*}
        E[<\vec{u},\vec{v}>] & =  \sum_k E[u_i v_i] & \text{By linearity of
        expectation}                                                               \\
                             & =\sum_k E[u_i]E[v_i] & \text{Assuming independence} \\
                             & = 0
    \end{align*}

    \begin{align*}
        E[(<\vec{u},\vec{v}>-E[<\vec{u},\vec{v}>])^2] & = E[(<\vec{u},\vec{v}>)^2]
        - E[<\vec{u},\vec{v}>]^2                                                                                                \\
                                                      & = E[(<\vec{u},\vec{v}>)^2]                                              \\
                                                      & =  \sum_k E[{u_i}^2] E[{v_i}^2] & \text{By linearity of expectation and
        indepedence}                                                                                                            \\
                                                      & = d_k
    \end{align*}

\fi
